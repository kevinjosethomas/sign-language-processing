\documentclass[../paper.tex]{subfiles}
\begin{document}

\section{Conclusion}
We have presented an open-source fingerspell recognition and semantic pose retrieval interface with the goal of advancing sign language translation systems. The interface combines convolutional neural networks and pose estimation models to create a system for translating between ASL fingerspelling and spoken English. The fingerspell recognition module translates ASL fingerspelling into spoken English, while the semantic pose retrieval module converts spoken English into ASL poses. The interface's ability to function reliably in real-time and under diverse environmental conditions, such as varying skin tones, backgrounds, and hand sizes, is an important step towards making sign language translation more accessible and inclusive. This adaptability ensures that the technology can be effectively used by a broader audience without the need for specialized environments or equipment, which is crucial for real-world applications.

Moreover, the modular design of the interface allows for easy integration into various platforms, making it a versatile tool for enhancing accessibility and inclusivity in a wide range of existing applications. We hope that this project will invite further collaboration and innovation in the field of sign language translation, inspiring developers to build upon the existing system and create more advanced solutions that capture the richness and complexity of sign language communication. By engaging with the Deaf community and incorporating their feedback, we can continue to improve the interface and ensure that it meets the diverse needs of its users.

While the current system provides a solid foundation for sign language translation, it more importantly serves as a stepping stone towards more advanced ASL translation systems. Fingerspelling is only one small aspect of ASL, and there is much more work to be done to capture the full richness and complexity of sign language grammar and syntax. By building upon this foundation, we can create more sophisticated systems that capture the nuances of sign language communication and provide a more inclusive and accessible experience for all individuals.

\end{document}